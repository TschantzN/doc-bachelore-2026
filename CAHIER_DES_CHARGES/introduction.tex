\section{Introduction}
Le pilotage de drones FPV )First Person View) requiert un retour vidéo en temps 
réel très fiable. Actuellement, la majorité de ces systèmes repose sur des liaisons 
analogiques ou numériques fonctionnant dans la bande des 5.8 GHz. Bien que cette bande offre une 
large bande passante, elle a une faible capacité de pénétration des obstacles (Non-Line-Of-Sight) et
donc une portée limitée en environnement urbain ou forestier.

La norme IEEE 802.11ah ou Wi-Fi HaLow, quant à elle utilise une bande 
sub-GHz (inférieur à 1 GHz). Elle permet théoriquement une portée kilométrique et une bien meilleure pénétration des 
matériaux. Cependant, le Wi-Fi standard n'étant initialement pas conçu pour le flux vidéo 
déterministe, des adaptations de la couche physique (PHY) et de la couche de liaison (MAC) sont 
nécessaires pour minimiser la latence.

Ce Travail de Bachelor a pour objectif de concevoir, développer et valider un prototype de transmission 
vidéo basé sur des modules Morse Micro. L'enjeu principal réside dans l'optimisation de la 
liaison radio, notamment par le forçage des schémas de modulation (MCS) et l'activation du codage correcteur 
d'erreurs LDPC (Low-Density Parity-Check).

\subsection{Références normatives}
Les documents suivants cités dans le texte constituent, pour tout ou partie de leur contenu, des 
exigences ou des bases théoriques fondamentales du présent projet.

\vspace{0.5cm}

\begin{longtable}{|p{4.5cm}|p{10.5cm}|}
\caption{Références normatives}
\label{tab:normes} \\
\hline
\textbf{Norme / Standard} & \textbf{Description} \\
\hline
\endfirsthead

\multicolumn{2}{c}%
{{\bfseries \tablename\ \thetable{} -- suite de la page précédente}} \\
\hline
\textbf{Norme / Standard} & \textbf{Description} \\
\hline
\endhead

\hline \multicolumn{2}{|r|}{{Suite à la page suivante}} \\ \hline
\endfoot

\hline
\endlastfoot

\textbf{IEEE 802.11ah-2016} & Amendement au standard IEEE 802.11 définissant les spécifications des couches PHY 
et MAC pour le fonctionnement dans les bandes exemptes de licence sous 1 GHz (Wi-Fi HaLow). \\
\hline
\textbf{IETF RFC 768} & Spécification du protocole UDP (User Datagram Protocol), utilisé pour la transmission vidéo à faible latence. \\
\hline
\textbf{ITU-T H.264 / H.265} & (Si applicable) Normes de codage vidéo très efficace utilisées par l'encodeur 
vidéo embarqué sur le drone. \\
\end{longtable}

\newpage

\subsection{Abréviations}
Cette section précise les terminologies techniques et les acronymes propres au domaine des télécommunications, des systèmes 
embarqués et du projet.

\vspace{0.5cm}

\begin{longtable}{|l|p{12cm}|}
\caption{Liste des abréviations}
\label{tab:abreviations} \\
\hline
\textbf{Terme} & \textbf{Définition} \\
\hline
\endfirsthead

\multicolumn{2}{c}%
{{\bfseries \tablename\ \thetable{} -- suite de la page précédente}} \\
\hline
\textbf{Terme} & \textbf{Définition} \\
\hline
\endhead

\hline \multicolumn{2}{|r|}{{Suite à la page suivante}} \\ \hline
\endfoot

\hline
\endlastfoot

\textbf{BCC} & Binary Convolutional Code (Code convolutif binaire) - Méthode de correction d'erreur standard en Wi-Fi. \\
\hline
\textbf{FPV} & First Person View (Vol en immersion) - Pilotage via une caméra embarquée retransmettant la vidéo en temps réel au pilote. \\
\hline
\textbf{HEIG-VD} & Haute École d'Ingénierie et de Gestion du Canton de Vaud. \\
\hline
\textbf{IEEE} & Institute of Electrical and Electronics Engineers. \\
\hline
\textbf{LDPC} & Low-Density Parity-Check (Code de contrôle de parité à faible densité) - Algorithme avancé de correction d'erreurs offrant une meilleure robustesse que le BCC. \\
\hline
\textbf{MAC} & Media Access Control (Contrôle d'accès au support) - Sous-couche de liaison de données. \\
\hline
\textbf{MCS} & Modulation and Coding Scheme (Schéma de modulation et de codage) - Indice définissant le débit de données physique en fonction de la modulation et du taux de codage (code rate). \\
\hline
\textbf{MVP} & Minimum Viable Product (Produit Minimum Viable) - Version fonctionnelle de base garantissant la validation du concept technique. \\
\hline
\textbf{NLOS} & Non-Line-Of-Sight (Absence de ligne de vue) - Situation où un obstacle sépare l'émetteur du récepteur. \\
\hline
\textbf{OS} & Operating System (Système d'exploitation) \\
\hline
\textbf{PHY} & Physical Layer (Couche physique) - Première couche du modèle OSI. \\
\hline
\textbf{RTOS} & Real-Time Operating System (Système d'exploitation temps réel) - Système conçu pour traiter les événements avec des contraintes temporelles strictes (utilisé sur le RX). \\
\hline
\textbf{RX} & Receiver (Récepteur) - Station au sol. \\
\hline
\textbf{TX} & Transmitter (Émetteur) - Module embarqué sur le Drone. \\
\hline
\textbf{UDP} & User Datagram Protocol - Protocole de communication en mode non connecté, privilégié pour les transmission en temps réel. \\
\end{longtable}

