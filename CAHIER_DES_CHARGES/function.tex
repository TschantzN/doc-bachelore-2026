\section{Fonctions et exigences du système}
Pour pallier le manque de documentation et gérer la complexité, trois stratégies de 
communication ont été identifiées. Elles sont classées par ordre de préférence 
et font partie intégrante des exigences techniques du système.

\begin{itemize}
    \item \textbf{Option A (Privilégiée) :} Protocole UDP Broadcast côté TX (Jetson) 
    combiné à une réception en mode moniteur (Monitor Mode) via \texttt{libpcap} côté RX (RTOS).
    \item \textbf{Option B (Alternative performance) :} Diffusion brute (Raw Broadcast) 
    avec LDPC. C'est l'option la plus rapide mais la plus complexe techniquement.
    \item \textbf{Option C (Sécurité) :} Utilisation des exemples standards du 
    constructeur. Solution techniquement simple mais avec des performances potentiellement 
    trop faibles.
\end{itemize}

\vspace{0.5cm}

\begin{longtable}{|p{6cm}|p{9cm}|}
\caption{Fonctions de service et exigences du système à concevoir}
\label{tab:fonctions_service} \\
\hline
\textbf{Fonctions de service} & \textbf{Exigences} \\
\hline
\endfirsthead

\multicolumn{2}{c}%
{{\bfseries \tablename\ \thetable{} -- suite de la page précédente}} \\
\hline
\textbf{Fonctions de service} & \textbf{Exigences} \\
\hline
\endhead

\hline \multicolumn{2}{|r|}{{Suite à la page suivante}} \\ \hline
\endfoot

\hline
\endlastfoot

\textbf{FS 1 :} Transmettre des données sans fil (MVP) 
& \textbf{E 1 :} Utiliser la norme IEEE 802.11ah (Wi-Fi HaLow). \\
\cline{2-2}
& \textbf{E 2 :} Forcer l'utilisation d'un MCS bas (index 1 ou 2). \\
\cline{2-2}
& \textbf{E 3 :} Activer la correction d'erreur LDPC. \\
\hline
\textbf{FS 2 :} Transmettre un flux vidéo (Objectif Final) 
& \textbf{E 4 :} Interfacer l'encodeur vidéo avec la carte Jetson. \\
\cline{2-2}
& \textbf{E 5 :} Assurer le décodage et l'affichage sur la station de recéption. \\
\end{longtable}

\vspace{0.5cm}

\begin{longtable}{|p{6cm}|p{9cm}|}
\caption{Fonctions techniques et exigences du système à concevoir}
\label{tab:fonctions_techniques} \\
\hline
\textbf{Fonctions techniques} & \textbf{Exigences} \\
\hline
\endfirsthead

\multicolumn{2}{c}%
{{\bfseries \tablename\ \thetable{} -- suite de la page précédente}} \\
\hline
\textbf{Fonctions techniques} & \textbf{Exigences} \\
\hline
\endhead

\hline \multicolumn{2}{|r|}{{Suite à la page suivante}} \\ \hline
\endfoot

\hline
\endlastfoot

\textbf{FT 1 :} Assurer la communication 
& \textbf{E 6 :} Implémenter la communication selon l'Option A (UDP Broadcast / Monitor mode) en priorité. \\
\cline{2-2}
& \textbf{E 7 :} Les options B (Raw) et C (Exemples) serviront de plans secondaires. \\
\hline
\textbf{FT 2 :} S'adapter aux limitations matérielles 
& \textbf{E 8 :} L'architecture doit permettre la substitution du module TX (EKH19 par EKH05-01) en cas d'incompatibilité avec les modifications du pilote. \\
\end{longtable}

\vspace{0.5cm}

\begin{longtable}{|p{6cm}|p{9cm}|}
\caption{Fonctions de contrainte et exigences du système à concevoir}
\label{tab:fonctions_contrainte} \\
\hline
\textbf{Fonctions de contrainte} & \textbf{Exigences} \\
\hline
\endfirsthead

\multicolumn{2}{c}%
{{\bfseries \tablename\ \thetable{} -- suite de la page précédente}} \\
\hline
\textbf{Fonctions de contrainte} & \textbf{Exigences} \\
\hline
\endhead

\hline \multicolumn{2}{|r|}{{Suite à la page suivante}} \\ \hline
\endfoot

\hline
\endlastfoot

\textbf{FC 1 :} Respecter la contrainte temps 
& \textbf{E 9 :} Réalisable en 450 heures. Le MVP fait foi si l'objectif final (idéale) n'a pas pu être réalisé. \\
\hline
\textbf{FC 2 :} Composer avec la documentation limitée 
& \textbf{E 10 :} Justifier les choix techniques basés sur l'ingénierie inverse et les tests empiriques. \\
\end{longtable}