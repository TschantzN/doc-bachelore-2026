\section{Description du problème}
Un drone FPV nécessite un retour vidéo avec une latence très faible et une 
grande robustesse aux obstacles (pénétration NLOS). Le standard Wi-Fi HaLow 
(802.11ah) en bande Sub-1 GHz offre des caractéristiques qui le rende  
idéales pour ce cas d'usage, mais n'est pas nativement optimisé pour un flux 
continu à basse latence.


Le défi technique majeur de ce projet se trouve dans le contrôle de bas niveau 
des puces Morse Micro (MM8108). La documentation constructeur étant limitée, 
l'activation des paramètres auquels nous souhaitons accéder pour optimiser la portée 
(codage LDPC et Schéma de Modulation et de Codage MCS restreint à 1 ou 2) requiert une 
analyse du fonctionnement bas niveau de ces éléments et de leurs firmwares.
Impliquant donc des modifications directement dans les pilotes matériels.\newline

\textbf{Objectifs par paliers (Gestion du temps - 450h) :}\newline
Afin de garantir un résultat intéressant du projet malgrès les incertitudes matérielles et 
logicielles, les livrables sont divisés en deux niveaux :

\begin{itemize}
    \item \textbf{Objectif minimum (MVP) :}Établir l'architecture matérielle complète de bout en 
bout et prouver la transmission de paquets de données génériques (texte, fichiers) 
avec les contraintes physiques imposées (HaLow, MCS 1 ou 2, LDPC activé).
    \item \textbf{Objectif final (Idéal) :} Remplacer les données génériques par le flux vidéo réel 
de l'encodeur, en respectant les contraintes de basse latence jusqu'à l'affichage sur 
la station au sol.\newline
\end{itemize}

\textbf{Architecture matérielle flexible :}
\begin{itemize}
    \item \textbf{Émission (TX) :}Encodeur vidéo $\rightarrow$ NVIDIA Jetson (OS Linux) $\rightarrow$ 
Module Wi-Fi MM8108-EKH19 (Option de repli : EKH05-01 si l'EKH19 s'avère incompatible avec 
les modifications du pilote).
    \item \textbf{Réception (RX) :}Module Wi-Fi EKH05-01 (RTOS (FreeRTOS) pour minimiser la latence logicielle) 
$\rightarrow$ PC (Décodage et Affichage).
\end{itemize}